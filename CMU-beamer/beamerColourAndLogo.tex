\documentclass{beamer}
\usepackage[utf8]{inputenc}
\usepackage{default}
\usepackage{graphicx}
\usepackage{amsmath,amssymb}
\usepackage{mathtools}
\usepackage{algorithm}
\usepackage{algpseudocode}
\usepackage{subcaption}
\usepackage{textpos}

\usecolortheme{beaver}
\usefonttheme{default}

\hypersetup {
  bookmarks=false,
  pdfpagelabels=false,
  hyperfootnotes=false,
  hyperindex=false,
  pageanchor=false,
  colorlinks
}

\makeatletter
\let\saved@hyper@linkurl\hyper@linkurl
%\let\saved@hyper@linkfile\hyper@linkfile
\let\saved@hyper@link@\hyper@link@
\AtBeginDocument{%
  % Since the whole document is affected, only the \begin part of
  % environment `NoHyper' is needed.
  \NoHyper
  \let\hyper@linkurl\saved@hyper@linkurl % needed by \url
  %\let\hyper@linkfile\saved@hyper@linkfile % needed by \href{<file>}
  \let\hyper@link@\saved@hyper@link@ % needed by \href{<url>}
}
\makeatother

\renewcommand{\algorithmicrequire}{\textbf{Input:}}
\renewcommand{\algorithmicensure}{\textbf{Output:}}

\newcommand{\blue}[1]{\textcolor{blue}{#1}}
\newcommand{\RR}{\mathbb{R}}
\newcommand{\diag}{\text{diag}}
\newcommand\indep{\perp\!\!\!\perp}

\DeclareMathOperator*{\argmin}{arg\,min}
\DeclareMathOperator*{\argmax}{arg\,max}

% shows how to change default (blue) colours in the default beamer theme
% found here: http://joerglenhard.wordpress.com/tag/latex/
\definecolor{CMURed}{RGB}{145,11,46}
\setbeamercolor{section in toc}{fg=CMURed}
\setbeamercolor{title}{fg=CMURed}
\setbeamercolor{frametitle}{fg=CMURed}
\setbeamercolor{structure}{fg=CMURed}
\setbeamertemplate{footline}[page number]

% adds logo in the footer
\logo{\includegraphics[scale=.15]{cmu_logo_red}}

\title[]{Online Learning: A Survey on Regret Bound Analysis}
\author[Han Zhao]{Han Zhao \\ Machine Learning Department \\ han.zhao@cs.cmu.edu}
\institute{\includegraphics[scale=0.25]{cmu_logo_red}}
\date{{Apr. 28th, 2016}}


\begin{document}
\begin{frame}
	\titlepage
\end{frame}

\newcommand{\defeq}{\vcentcolon=}
\newcommand{\eqdef}{=\vcentcolon}
\newcommand{\Reg}{\text{Regret}}

\begin{frame}
\frametitle{Online Learning: A Survey on Regret Bound Analysis}
Online learning as a (adversarial) game:
\vspace*{-2pt}
\begin{algorithmic}[1]
\FOR {$t = 1, 2,\ldots$}
	\STATE	Receive an observation $\mathbf{x}_t\in\mathcal{X}$.
	\STATE 	Make a prediction $p_t\in\mathcal{Y}$.
	\STATE	Receive the true answer $y_t\in\mathcal{D}$.
	\STATE	Incur a loss $l(p_t, y_t)$.
\ENDFOR
\end{algorithmic}
Goal: Regret minimization
$$\Reg_T(\mathcal{A}, \mathcal{H})\defeq  \sum_{t=1}^T l(\mathcal{A}(\mathbf{x}_t), y_t) - \min_{h^*\in \mathcal{H}}\sum_{t=1}^T l(h^*(\mathbf{x}_t), y_t)$$

An algorithm is called no-regret w.r.t. $\mathcal{H}$ if $\Reg_T(\mathcal{A}, \mathcal{H}) = o(T)$.
\end{frame}

\end{document}

